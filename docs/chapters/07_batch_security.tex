% !TeX root = ../main.tex
% Add the above to each chapter to make compiling the PDF easier in some editors.
\chapter{Formalized Batch Version Security}\label{chapter:batch_security}
In this chapter, we outline the security properties for the batched version of the KZG as well as the idea behind the formalization of each security property. Again, as in chapter \ref{chapter:security}, we will not give concrete proofs in Isabelle syntax as this would go beyond the scope due to a lot of boilerplate Isabelle-specific abbreviations and functions.

Similarly to chapter \ref{chapter:security}, each section covers one security property. Firstly, we describe the paper proof, followed by the game-based proof. Lastly, we outline the formalization, specifically the security game, the reduction algorithm and if needed additional formalization details.

Note, the property of \textit{polynomial-binding} does not involve any partially opening functions (i.e. neither CreateWitness nor CreateWitnessBatch) and hence does not change for the batched version. The property thus holds automatically for the batched version.

\section{Evaluation Binding}
We formalize the property \textit{evaluation binding} slightly adapted from \ref{PCS_def} (we introduce \textit{VerifyEvalBatch}) as described in the original paper \parencite{KZG}. Formally we define \textit{evaluation binding} as the following game against an efficient adversary $\mathcal{A}$ according to \parencite{KZG}: 
\begin{equation*}
    \left(
        \begin{aligned}
            PK & \leftarrow \text{KeyGen}, \\
            (C, i,\phi_i,\omega_i, B, r(x),\omega_B) & \leftarrow \mathcal{A} \ PK, \\
            \_::\text{unit} & \leftarrow \text{assert\_spmf}(i\in B \land \phi_i \ne r(x)), \\
            b &= \text{VerifyEval}(PK, C, i, \phi_i, \omega_i),\\
            b' &= \text{VerifyEvalBatch}(PK, C, B, r(x), \omega_B),\\
            & : \ b \land b'
        \end{aligned}
        \right)
\end{equation*}
Intuitively this game expresses that no adversary can find an $r(x)$ (with a witness) and an evaluation (with a witness) that diverges from the evaluation of $r(x)$ at any $i \in B$.

\subsection*{Original Paper Proof}
The paper proof is similar to the evaluation binding proof outlined in \ref{security:binding:paper}. Essentially, the values provided by the adversary, if they are correct i.e. accepted by VerifyEval and respectively VerifyEvalBatch, can be rearranged to obtain the result (the secret key) for the t-BSDH instance, which is the public key. Nevertheless, we outline the paper proof concretely for completeness:

The paper proof argues that given an adversary $\mathcal{A}$, that can break the evaluation binding property, an algorithm $\mathcal{B}$ can be constructed, that can break the t-BSDH assumption \parencite{KZG}. The concrete construction for $\mathcal{B}$ is: given the t-BSDH instance $tsdh\_inst =(\mathbf{g}, \mathbf{g}^{\alpha}, \mathbf{g}^{\alpha^2},\dots, \mathbf{g}^{\alpha^t})$, call $\mathcal{A}$ with $tsdh\_inst$ as $PK$ for $(C,B, r(x), \omega_B, i \in B,\phi(i),\omega_i)$ and return: 
$$ \biggl(e\biggl(\mathbf{g}^{p'(\alpha)}, \omega_B\biggr) \oplus e\biggl(\frac{\mathbf{g}^{r'(\alpha)}}{\omega_i}, \mathbf{g}\biggr)\biggr)^{\frac{1}{\phi(i)-r(i)}}$$
where $p'(x)$ is the product polynomial $\prod_{j\in B\backslash\{i\}}^{}(x-j) = \frac{\prod_{j\in B}^{}(x-j)}{(x-i)}$ and $r'(x)=\frac{r(x)-r(i)}{(x-i)}$ \parencite{KZG}.
The reason to why $\mathcal{B}$ is a correct construction is the following: 

Breaking evaluation binding means, that $\mathcal{A}$, given a valid public key $PK$, can give a Commitment $C$ and two witness-tuples, $\langle i, \phi(i),\omega_i\rangle$ and $\langle B, r(x), \omega_B\rangle$, where $i \in B$, such that $VerifyEval(PK, C,\langle i,\phi(i), \omega_i\rangle )$ and $VerifyEvalBatch(PK, C,\langle B,r(x), \omega_B\rangle )$ return true \parencite{KZG}. Since \textit{VerifyEvalBatch}, as well as \textit{VerifyEval}, is a pairing check against $e(C,\mathbf{g})$ we can conclude that: 

$$e(\omega_i,\mathbf{g}^{\alpha-i})e(\mathbf{g}, \mathbf{g})^{\phi(i)} = e(C,\mathbf{g}) = e(\mathbf{g}^{\prod_{i\in B}^{}(\alpha-i)}, \omega_B)e(\mathbf{g}, \mathbf{g}^{r(\alpha)})$$
which is the pairing term for: 
\begin{equation*}
    \begin{aligned}
        &\psi_i(\alpha) \cdot (\alpha-i) + \phi(i) = \prod_{j\in B}^{}(\alpha-j) \cdot \psi_B(\alpha) + r(\alpha) \\
        \iff&\phi(i) - r(\alpha) = \prod_{j\in B}^{}(\alpha-j) \cdot \psi_B(\alpha) - \psi_i(\alpha) \cdot (\alpha-i)\\
        \iff&\phi(i) - r(\alpha) = \frac{\prod_{j\in B}^{}(\alpha-j)}{(\alpha-i)} \cdot (\alpha-i) \cdot \psi_B(\alpha) - \psi_i(\alpha) \cdot (\alpha-i)\\
        \iff&\phi(i) - r(\alpha) = p'(\alpha) \cdot (\alpha-i) \cdot \psi_B(\alpha) - \psi_i(\alpha) \cdot (\alpha-i)\\
        \iff&\phi(i) - r(\alpha) = (p'(\alpha) \cdot \psi_B(\alpha) - \psi_i(\alpha)) \cdot (\alpha-i)\\
        \iff&\phi(i) - (r'(\alpha) \cdot (\alpha-i) + r(i)) = (p'(\alpha) \cdot \psi_B(\alpha) - \psi_i(\alpha)) \cdot (\alpha-i)\\
        \iff&\phi(i) - r(i) = (p'(\alpha) \cdot \psi_B(\alpha) - \psi_i(\alpha) + r'(\alpha)) \cdot (\alpha-i)\\
        \iff&\frac{1}{(\alpha-i)} = \frac{p'(\alpha) \cdot \psi_B(\alpha) - \psi_i(\alpha) + r'(\alpha)}{\phi(i) - r(i)}\\
    \end{aligned}
\end{equation*}
where $\psi_i(\alpha)= log_{\mathbf{g}}\ \omega_i$ and $\psi_B(\alpha)= log_{\mathbf{g}}\ \omega_B$ and omitting the $e(C,\mathbf{g})$
\parencite{KZG}.

Hence: 
\begin{equation*}
    \begin{aligned}
        &\biggl(e\biggl(\mathbf{g}^{p'(\alpha)}, \omega_B\biggr) \oplus e\biggl(\frac{\mathbf{g}^{r'(\alpha)}}{\omega_i}, \mathbf{g}\biggr)\biggr)^{\frac{1}{\phi(i)-r(i)}} \\
        &= \biggl(e\biggl(\mathbf{g}^{p'(\alpha)}, \mathbf{g}^{\psi_B(\alpha)}\biggr) \oplus e\biggl(\frac{\mathbf{g}^{r'(\alpha)}}{\mathbf{g}^{\psi_i(\alpha)}}, \mathbf{g}\biggr)\biggr)^{\frac{1}{\phi(i)-r(i)}} \\
        &= \biggl(e(\mathbf{g}, \mathbf{g})^{p'(\alpha) \cdot \psi_B(\alpha)} \oplus e(\mathbf{g}, \mathbf{g})^{r'(\alpha) - \psi_i(\alpha)}\biggr)^{\frac{1}{\phi(i)-r(i)}}\\
        &= \biggl(e(\mathbf{g}, \mathbf{g})^{p'(\alpha) \cdot \psi_B(\alpha) + r'(\alpha) - \psi_i(\alpha)} \biggr)^{\frac{1}{\phi(i)-r(i)}}\\
        &= e(\mathbf{g}, \mathbf{g})^{\frac{p'(\alpha) \cdot \psi_B(\alpha) + r'(\alpha) - \psi_i(\alpha)}{\phi(i)-r(i)}}\\
        &= e(\mathbf{g}, \mathbf{g})^{\frac{1}{\alpha-i}}\\
    \end{aligned}
\end{equation*}
Since $e(\mathbf{g},\mathbf{g})^{\frac{1}{\alpha-i}}$ breaks the t-BSDH assumption for i, $\mathcal{B}$ is correct \parencite*{KZG}.

\subsection*{Game-based Proof}
The transformation into a game-based proof is analog to \ref{security:binding:gamebased}: 

The goal of this proof is to show the following theorem, which states that the probability of any adversary breaking evaluation binding is less than or equal to winning the DL game (using the reduction adversary) in a game-based proof:
\begin{lstlisting}[language=isabelle]
    theorem batchOpening_binding: "bind_advantage $\mathcal{A}$ 
        $\le$ t_BSDH.advantage (bind_reduction $\mathcal{A}$)"
\end{lstlisting}

A look at the \textit{evaluation binding} and the \textit{t-BSDH} games reveals that \textit{KeyGen}'s generation of PK is equivalent to generating a t-BSDH instance. Furthermore, the games differ only in their checks in the respective \textit{assert\_spmf} calls (and the adversary's return types).  
Additionally, we know from the paper proof that the adversary's messages, if correct and wellformed, which is checked in the eval\_bind game's asserts, already break the t-BSDH assumptions on PK. 
Hence we give the following idea (which is analog to \ref{security:binding:gamebased}) for the proof:
\begin{enumerate}
    \item rearrange the eval\_bind game to accumulate (i.e. conjuncture) the return-check and all other checks into an assert
    \item derive that this conjuncture of statements already implies that the t-BSDH is broken and add that fact to the conjuncture.
    \item erase every check in the conjuncture by over-estimation, to be only left with the result that the t-BSDH is broken.
\end{enumerate}
The resulting game is the t-BSDH game with the reduction adversary. 
See the Isabelle theory \textit{KZG\_BatchOpening\_bind} for the full formal proof.



\subsection*{Formalization}
We formally define the evaluation binding game in CryptHOL as follows:
\begin{lstlisting}[language=isabelle]
    TRY do {
        PK $\leftarrow$ KeyGen;
        (C, i, $\phi$_i, w_i, B, w_B, r_x) $\leftarrow \mathcal{A}$ PK;
        _::unit $\leftarrow$ assert_spmf($i\in B$ $\land$ $\phi$_i $\ne$ poly r_x i 
            $\land$ valid_msg $\phi$_i w_i $\land$ valid_batch_msg r_x w_B B);
        let b = VerifyEval PK C i $\phi$_i w_i;
        let b' = VerifyEvalBatch PK C B r_x w_B;
        return_spmf (b $\land$ b')
    } ELSE return_spmf False
\end{lstlisting}
The game captures the spmf over True and False, which represent the events that the adversary has broken evaluation binding or not.
The public key $PK$ is generated using the formalized \textit{KeyGen} function of the KZG. The adversary $\mathcal{A}$, given PK, outputs values to break the evaluation binding game, namely a commitment value $C$, a set of positions $B$, that are to be opened, a polynomial $r\_x$ which evaluates to the claimed evaluations of $\phi$ (the polynomial, $C$ is the commitment to) at the positions in B, a witness $w\_B$ that validates that the points $(i,r\_x(i))$ for all $i \in B$ are valid for $C$, a point $i \in B$, a claimed value $\phi\_i$ that should be different from $r\_x(i)$ and a witness, w\_i for the point $(i,\phi\_i)$. 
Note that we use \textit{assert\_spmf} to ensure that the adversary's messages are wellformed and correct, where correct means that $i \in B$ and $\phi$\_i and the evaluation of $r\_x$ at $i$ are indeed two different values. Should the assert not hold, the game is counted as lost for the adversary.
We assign to b and b' the result of the formalized \textit{VerifyEvalBatch} and respectively \textit{VerifyEval} algorithm of the KZG. Evaluation binding is broken if and only if b and b' hold i.e. both witnesses and verifying checks pass and thus the same commitment can efficiently be resolved to two different values at some point.

We formally define the reduction adversary as follows:
\begin{lstlisting}[language=isabelle]
    fun bind_reduction $\mathcal{A}$ PK = do {
        (C, i, $\phi$_i, w_i, B, w_B, r_x) $\leftarrow$ $\mathcal{A}$ PK;
        let p' = g_pow_PK_Prod PK (prod ($\lambda$i. [:-i,1:]) B div [:-i,1:]);
        let r' = g_pow_PK_Prod PK ((r_x - [:poly r_x i:]) div [:-i,1:]);
        return_spmf (-i, (e p' w_B $\oplus$ e (r' $\div$ w_i) $\mathbf{g}$)^(1/($\phi$_i - poly r_x i)))
    }
\end{lstlisting}
That is a higher-order function, that takes the evaluation binding adversary $\mathcal{A}$ and returns an adversary for the t-BSDH game.
That is, the function that calls the adversary $\mathcal{A}$ on some public key PK and returns $\bigl(-i, \bigl(e\bigl(\mathbf{g}^{p'(\alpha)}, \omega_B\bigr) \oplus e\bigl(\frac{\mathbf{g}^{r'(\alpha)}}{\omega_i}, \mathbf{g}\bigr)\bigr)^{\frac{1}{\phi(i)-r(i)}}\bigr)$, the solution to the t-BSDH game for $-i$. 

The term \textit{[:poly r\_x i:]} is Isabelle's notation for a constant polynomial with the constant value of the polynomial $r\_x$ evaluated at $i$. The term $[:-i,1:]$ is Isabelle's notation for the polynomial $(x-i)$ and \textit{prod ($\lambda$. [:-i,1:]) B} is the product term $\prod_{i\in B}^{}(x-i)$. The $g\_pow\_PK\_Prod$ function returns the group generator $\mathbf{g}$, exponentiated with the evaluation of a polynomial on the secret key $\alpha$, hence, $p'=\mathbf{g}^{\frac{\prod_{i\in B}^{}(x-i)}{(x-i)}}$ and $r'=\mathbf{g}^{\frac{r_x-r_x(i)}{(x-i)}}$.

\section{Hiding}
\label{batch:security:hiding}

We discuss our thoughts on extending the hiding property shown for the KZG in \ref*{security:hiding} to the batched version. We present two possible approaches to proving this property and outline why we think neither of them will work. 

Note, that the definition of hiding for the batched version does not unconditionally fulfill the \textit{hiding} property as defined in \ref{hiding_def} but requires additionally the polynomial to be uniformly randomly chosen to hold.

Formally we define \textit{hiding} as the following game against an efficient adversary $\mathcal{A}$, where $B \subset \mathbb{Z}_p$ is an arbitrary set of size t:
\begin{equation*}
    \left(
        \begin{aligned}
            \phi & \overset{{\scriptscriptstyle\$}}{\leftarrow} \{\phi. \text{ degree}(\phi)\le t\},\\
            PK & \leftarrow \text{KeyGen}, \\
            C & = \text{Commit}(PK,\phi), \\
            wtns\_tpl &= \text{CreateWitnessBatch}(PK,\phi,B),\\
            \phi' & \leftarrow \mathcal{A}(PK,C,wtns\_tpl), \\
            & : \ \phi = \phi'
        \end{aligned}
    \right)
\end{equation*}

\subsection*{Original Paper Proof}
The original paper \parencite{KZG} does not provide a proof for his property. 
The authors argue that the function \textit{CreateWitnessBatch} does not reveal any more information than \textit{CreateWitness} and thus the property is already shown by the hiding proof for the standard KZG \parencite{KZG}. 
However, this argument is not formally satisfying, which is why we discuss some approaches to proving this property in the following section.

\subsection*{Game-based Proof}
The goal of the hiding proof is to show the following theorem, which states that the probability of any adversary breaking hiding is less than or equal to winning the DL game (using the reduction adversary) in a game-based proof:
\begin{lstlisting}[language=isabelle]
    theorem batch_hiding: "batch_hiding_advantage $\mathcal{A}$ 
        $\le$ t_SDH.advantage (reduction $\mathcal{A}$)"
\end{lstlisting}

We focus on two approaches to proving the hiding property for the batch version: 
\begin{enumerate}
    \item reducing the batched hiding game to the normal hiding game 
    \item adapting the normal hiding proof to the batched version
\end{enumerate}
Firstly, we argue why we do not think that the hiding game for the batched version can be reduced to the hiding game of the normal KZG. We outline specifically the difficulties in transitioning from \textit{CreateWitnessBatch} to \textit{CreateWitness}.
Secondly, we argue why the hiding proof for the standard KZG cannot be trivially adapted to the batched version.

\subsubsection*{Reducing Batched Hiding to Normal Hiding}
For this approach, we aim to show the following theorem: 
\begin{lstlisting}[language=isabelle]
    theorem batch_hiding_reduction: "batch_hiding_advantage $\mathcal{A}$ 
        $\le$ hiding_advantage (hiding_reduction $\mathcal{A}$)"
\end{lstlisting}

Essentially, this boils down to creating a reduction adversary \textit{hiding\_reduction} that wins the batch hiding game using the adversary for the normal hiding game. Hence emulating the normal hiding game in the reduction adversary.

Note, that the majority of the inputs for the adversaries are equal, they take firstly the public key $PK$ and secondly a commitment $C$. The only value they differ in is the witness tuple. Specifically, the normal hiding game computes the $|I|$ many witnesses for the adversary as '$\text{map } (\text{CreateWitness}(PK,\phi,i))\ I$', while the batched version computes one witness tuple as '$\text{CreateWitnessBatch}(PK,\phi,B)$' for its respective adversary.

Hence, $\text{CreateWitness}(PK,\phi,i)$ would need to be emulated using only the reduction adversary's inputs to emulate the normal hiding game in the reduction adversary for the batched version. 
One would need to emulate $\text{CreateWitness}(PK,\phi,i)$ for all $i \in I$ using only $PK$, $C$, and the result from $\text{CreateWitnessBatch}(PK,\phi,B)$.

As a little reminder, the result from \textit{CreateWitness} is a tuple $(i, \phi(i), \omega_i)$, where $\omega_i$ is the wintess for the point $(i,\phi(i))$. The result from \textit{CreateWitnessBatch} is a tuple $(B, r(x), \omega_B)$, where B is a set of values, $r(i)=\phi(i)$ for all $i\in B$, and $\omega_B$ is a witness for all points $(i,r(i))$ for $i\in B$.
Furthemore, note $\omega_i = \mathbf{g}^{\psi_i(\alpha)}$ where $\psi_i(x)=\frac{\phi(x)-\phi(i)}{(x-i)}$ and $\omega_B = \mathbf{g}^{\psi_B(\alpha)}$ where $\psi_B(x)=\frac{\phi(x)-r(x)}{\prod_{i \in B}^{}(x-i)}$ and $r(x)=\phi(x) \text{ mod } \prod_{i \in B}^{}(x-i)$.

Converting $(B,r(x))$ in $|B|$ many points $(i,\phi(i))$ for $i \in B$ is trivial.
The challenge is to compute $\mathbf{g}^{\psi_i(\alpha)}$ from $PK,C,B,r(x),$ and $\mathbf{g}^{\psi_B(\alpha)}$.

To naively compute $\mathbf{g}^{\psi_i(\alpha)}=\mathbf{g}^{\frac{\phi(\alpha)-\phi(i)}{(\alpha-i)}}$ one would need to know the value of $(\alpha-i)$ and the group values for $\phi(\alpha)$ and $\phi(i)$. Note, that we can compose group values for addition and subtraction in the exponents, but no multiplication or division. While we do know the group values of $\phi(\alpha)$ and $\phi(i)$ from $C=\mathbf{g}^{\phi(\alpha)}$ and $r(i)=\phi(i)$, we do not know the value of $(\alpha-i)$. The scheme reveals only group values dependent on $\alpha$ (the commitment $C$, the witness $\omega_B$ and the public key $PK$), but no field values, thus $\alpha$ is secure by the DL assumption. Hence, constructing a field value dependent on $\alpha$, like $\alpha-i$, from group values is also impossibly hard by DL assumption.

One could ideate to somehow carve $\psi_i$ out of $\psi_B$, as $\psi_B=\mathbf{g}^{\frac{\phi(\alpha)-r(\alpha)}{\prod_{i \in B}^{}(\alpha-i)}}$ already includes the divions by $\alpha-i$. However, this would require knowledge of $\prod_{i \in B\backslash\{i\}}^{}(\alpha-i)$, which again is a field element that dependents on $\alpha$. Hence this approach is also impossible by DL assumption.

Thus we do not think that the batched hiding game can be reduced to the normal hiding game.

\subsubsection*{Adapting the Normal Hiding Proof}
For this approach, we aim to show the following theorem, which is the one firstly outlined in this section: 
\begin{lstlisting}[language=isabelle]
    theorem batch_hiding: "batch_hiding_advantage $\mathcal{A}$ 
        $\le$ t_SDH.advantage (reduction $\mathcal{A}$)"
\end{lstlisting}

Note, that the batched hiding game is equal to the normal hiding game except for the computation of the witnesses. While the normal hiding game computes them as 
'$\text{map } (\text{CreateWitness}(PK,\phi,i))\ I$', the batched version computes them as \\
'$\text{CreateWitnessBatch}(PK,\phi,B)$'. 

The normal reduction proof, outlined in \ref*{security:hiding:game_based_transf}, emulates $CreateWitness$ without knowing the full polynomial. Only $t$ points of the polynomial and one additional group point (i.e. $(i,\mathbf{g}^{\phi(i)})$) are known. Still $t$ correct points $(i,\phi(i))$ can be outputted as part of the result of $CreateWitness$, even though the full polynomial is unknown.

However, $CreateWitnessBatch$ does not output $t$ points for the evaluations, but the remainder $r(x)=\phi(x) \text{ mod } \prod_{i \in B}^{}(x-i)$. Since the $t+1$th group point is hidden by DL assumption, the full polynomial $\phi(x)$ cannot be known and thus the remainder is not trivially computable. Hence, the result of $CreateWitnessBatch$ cannot be trivially emulated. Thus the hiding proof from \ref*{security:hiding:game_based_transf} is not trivially adaptable to the batched hiding game.
Furthermore, note that all information about $\phi(x)$, like evaluations, except for the $t$ known points, are provided in a grouped manner i.e. as $\mathbf{g}^{\phi(\alpha)}$. Thus intuitively we do not think it is possible to derive a field element-based value, like the remainder $r(x)$, that depends on more information about $\phi$ than the $t$ known points due to the DL assumption.

For now, it remains an open question whether this property can be proven or not.


\section{Knowledge Soundness}
We extend the knowledge soundness property as defined in \ref{knowledgesound_def}
and proved for the KZG in \ref{security:knowledgesound} to the batched KZG version. Formally we define \textit{knowledge soundness} as the following game against an efficient AGM adversary $\mathcal{A=(A',A'')}$ and an efficient extractor $E$: 
\begin{equation*}
    \left(
        \begin{aligned}
            PK &\leftarrow \text{KeyGen}, \\
            (C,calc\_vec, \sigma) &\leftarrow \mathcal{A'}, \\
            \_::\text{unit} &\leftarrow \text{assert\_spmf}\biggl(\text{len}(PK)=\text{len}(calc\_vec) \land C = \prod_{1}^{len(calc\_vec)}PK_i^{calc\_vec_i}\biggr), \\
            \phi &= E(C, calc\_vec),\\
            (i, B, r(x), \omega_B) &\leftarrow \mathcal{A''}(\sigma, PK, C, calc\_vec), \\
            & : \ \phi(i) \ne r(i) \land \text{VerifyEvalBatch}(PK,C,B,r(x),\omega_B)
        \end{aligned}
        \right)
\end{equation*}
We omit the AGM vector for \textit{w\_B} as we do not need it for our proof, for completeness one can think of it as an implicit output that is never used.

\subsection*{Original Paper Proof}
\label{batch:security:knowledge:paper}
The proof is analog to \ref{security:knowledge:paper}:

Firstly, note that $calc\_vec$ provides exactly the coefficients one would need to obtain from a polynomial one wants to commit to. Thus $E$ can return a polynomial $\phi$ that has exactly the coefficients from $calc\_vec$. Since we know $C= \text{Commit}(PK, \phi)$ and the KZG is correct, we can conclude that for every value $i$, VerifyEval$(PK, C, i, \phi, \phi(i))$ must hold. Hence, if the adversary $\mathcal{A''}$ can provide $B$ and $r(x)$, such that $i\in B$, $r(i)\ne\phi(i)$
and VerifyEvalBatch$(PK, C, B, r(x), \omega_B)$ hold, the evaluation binding property is already broken because VerifyEval and VerifyEvalBatch verify for two different evaluations at the same point of a polynomial.

\subsection*{Game-based Proof}
\label{batch:security:knowledge:gamebased}

The game-based transformation is analog to \ref{security:knowledge:gamebased}:

The goal of this proof is to show the following theorem, which states that the probability of any efficient AGM adversary breaking  knowledge soundness is less than or equal to breaking the evaluation binding property (using a reduction adversary):

\begin{lstlisting}[language=isabelle]
    theorem knowledge_soundness_game_eq_bind_game_knowledge_soundness_reduction: 
        "knowledge_soundness_game E $\mathcal{A'}$ $\mathcal{A''}$ 
         = bind_game (knowledge_soundness_reduction E $\mathcal{A'}$ $\mathcal{A''}$)"
\end{lstlisting}
Moreover, since we already formalized the theorem \textit{evaluation\_binding} (i.e. that the probability of breaking evaluation binding is less than or equal to breaking the t-BSDH assumption), we get the following theorem through transitivity, given the $knowledge\_soundness\_game\_eq\_bind\_game_knowledge\_soundness\_reduction$ theorem holds: 

\begin{lstlisting}[language=isabelle]
    theorem knowledge_soundness: "knowledge_soundness_advantage $\mathcal{A}$ 
        $\le$ t_BSDH.advantage (bind_reduction (knowledge_soundness_reduction $\mathcal{A}$))"
\end{lstlisting}

Based on the idea from \ref{batch:security:knowledge:paper}, we formally define the following reduction adversary to the evaluation binding game, given the knowledge soundness adversary $\mathcal{A=(A',A'')}$, the extractor $E$, and the input for the evaluation binding adversary; the public key $PK$:
\begin{equation*}
    \left(
        \begin{aligned}
            (C,calc\_vec, \sigma) &\leftarrow \mathcal{A'}(PK), \\
            \phi &= E(C, calc\_vec),\\
            (i, B, r(x), \omega_B) &\leftarrow \mathcal{A''}(\sigma, PK, C, calc\_vec), \\
            \phi_i &= \phi(i), \\
            \omega_i &= \text{CreateWitness}(PK,\phi, i), \\
            &  (C, i, \phi_i, \omega_i, B, r(x), \omega_B)
        \end{aligned}
        \right)
\end{equation*}
The reduction creates a tuple $(C, i, \phi_i, \omega_i, B, r(x), \omega_B)$ to break the evaluation binding property. The commitment $C$ is determined by the adversary $\mathcal{A'}$, from which messages the extractor $E$ also computes the polynomial $\phi$, to which $C$ is a commitment (see \ref{security:knowledge:paper}). The adversary $\mathcal{A'}$ provides a set of positions $B$, the claimed evaluations of $\phi(x)$ to the positions in $B$ captured in the polynomial $r(x)$ (i.e. $\forall i\in B.\ r(i)\stackrel{!}{=}\phi(i)$), a witness $\omega_B$ for $r(x)$ and $B$ (i.e. such that $VerifyEvalBatch(PK,C,B,r(x), \omega_B)$ holds), and a position $i \in B$ for that it claims that $\phi(i) \ne r(i)$. Then the real evaluation of $\phi$ on $i$, $\phi(i)$ is computed as $\phi_i$ and a witness $\omega_i$ for the point $(i,\phi(i))$ is computed using $CreateWitness$.
Note, if the knowledge soundness adversary is correct, then $\phi_i=\phi(i)\ne r(x)$ and $\text{VerifyEval}(PK,C,i,\phi_i,\omega_i) \land \text{VerifyEvalBatch}(PK,C,B, r(x),\omega_B)$ holds. Hence the reduction is a correct and efficient adversary for \text{evaluation binding}.

For the game-based proof note that the knowledge soundness game and the evaluation binding game with the reduction adversary are equivalent except for the asserts: while the knowledge soundness game asserts 
$$\phi(i) \ne r(i) \land \text{VerifyEvalBatch}(PK, C, B,r(x),\omega_B)$$ the evaluation binding game asserts 
$$\phi_i \ne r(i) \land \text{VerifyEval}(PK, C, i, \phi_i, \omega_i) \land \text{VerifyEvalBatch}(PK, C, B, r(x), \omega_B)$$ where $\phi(i)=\phi_i$. Furthermore, note that the two statements are equivalent since VerifyEval for $\phi(i)$ is trivially true. Hence, the game-based proof is effectively equational reasoning over the asserts. The complete game-based proof is to be found in the Isabelle theory $KZG\_BatchOpening\_knowledge\_soundness$.

\subsection*{Formalization}
We formalize the knowledge soundness game in CryptHOL as follows: 
\begin{lstlisting}[language=isabelle]
    TRY do {
        PK $\leftarrow$ KeyGen;
        (C, calc_vec) $\leftarrow \mathcal{A'}$ PK;
        _ :: unit $\leftarrow$ assert_spmf (length PK = length calc_vec 
            $\land$ C = fold ($\lambda$ i acc. acc $\cdot$ PK!i ^ (calc_vec!i)) [0..<length PK] 1);
        let $\phi$ = E C calc_vec;
        (i, B, r_x, w_B) $\leftarrow \mathcal{A}$ PK C calc_vec;
        _::unit $\leftarrow$ assert_spmf(i$\in$B $\land$ valid_batch_msg r_x w_B B);
        return_spmf (VerifyEvalBatch PK C B r_x w_B $\land$  poly r_x i $\ne$ poly $\phi$ i)
    } ELSE return_spmf False
\end{lstlisting}
The game captures the spmf over True and False, which represent the events that the adversary has broken knowledge soundness or not.
Firstly, the public key $PK$ is generated using \textit{KeyGen}. 
Secondly, the adversary $\mathcal{A'}$ provides a commitment $C$ in the algebraic group model i.e. with the vector $calc\_vec$, which constructs $C$ from $PK$. We use an assert to ensure in Isabelle that the message of the $\mathcal{A'}$ is correct according to the AGM.
Thirdly, the extractor $E$ computes a polynomial $\phi$ given access to the messages of $\mathcal{A'}$. 
The second part of the adversary, $\mathcal{A''}$, computes a set of position $B$, an evaluation polynomial $r\_x$ that captures the claimed evaluations of $\phi$ on the positions $B$ (i.e. $\forall i\in B.\ \phi(i)\stackrel{!}{=}r(i)$), a witness $w\_B$ for $B$ and $r\_x$, and a position $i\in B$ for which $\phi(i)\ne r(i)$ should hold. We use an assert to check that the messages of $\mathcal{A''}$ are valid, that is to ensure in Isabelle that $w\_B$ is a group element and $i\in B$.
The adversary wins the game if and only if $r(x) \ne \phi(i)$ and VerifyEvalBatch($PK,C,B, r\_x, w\_B$) hold.

We formalize the reduction adversary to the binding game as defined in \ref{security:knowledge:gamebased}: 
\begin{lstlisting}[language=isabelle]
    fun reduction $\mathcal{(A',A'')}$ PK = do {
        (C, calc_vec) $\leftarrow$ $\mathcal{A'} PK$;
        _ :: unit $\leftarrow$ assert_spmf (length PK = length calc_vec 
            $\land$ C = fold ($\lambda$ i acc. acc $\cdot$ PK!i ^ (calc_vec!i)) [0..<length PK] 1);
        let $\phi$ = E C calc_vec;
        (i,$\phi$_i, w_i) $\leftarrow \mathcal{A}$ PK C calc_vec;
        _::unit $\leftarrow$ assert_spmf(valid_msg $\phi_i$, w_i);
        return_spmf (C, i, $\phi$_i, w_i, $\phi$_i',w_i')
    }
\end{lstlisting}
This function mirrors exactly the outline in \ref{batch:security:knowledge:gamebased} except for the validity checks of the adversary's messages in the form of the asserts (see the game definition above for details), thus we skip a detailed description.